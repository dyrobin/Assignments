\documentclass{article}
\usepackage[top=2cm, bottom=2.33cm, left=2.33cm, right=2.33cm]{geometry}
\usepackage[section]{placeins}
\usepackage{color}
\definecolor{aaltoYellow}{RGB}{254,203,0}
\newcommand{\TODO}[1]{\noindent\colorbox{aaltoYellow}{\color{black} TODO: #1}}
\usepackage[T1]{fontenc}
\usepackage[utf8]{inputenc}
\usepackage{lmodern}


\title{\vspace{-10pt}Exam Sheet}
\author{T-110.5150 - Applications and Services in Internet}
\date{Deadline: 12th December 2015, 23:59 Helsinki Time}

\begin{document}

\maketitle

\section*{Grading}
\noindent
The total number of points is 34.
You need at least 12 points to pass the exam.

\noindent
Grade boundaries:
\vskip 10pt

fail --- $<$ 12 points.

1 --- $\ge$ 12 points.

2 --- $\ge$ 17 points.

3 --- $\ge$ 22 points.

4 --- $\ge$ 26 points.

5 --- $\ge$ 30 points.

\section*{Questions}

\begin{enumerate}

\item Give a taxonomy of peer-to-peer (P2P) networks. Compare the advantages and disadvantages of each category. Then describe the strengths and weaknesses of P2P technology as a whole (from both technical and non-technical perspectives). (4 points)

\item Describe the approaches to improve the energy efficiency of mobile platforms. (4 points)

\item Describe the techniques to distribute content and strategies to stream multimedia. (4 points)

\item Describe enabling technologies and protocols for Internet of Things (IoT). (4 points)

\item (1) Describe the indoor localization technologies, and compare their strengths and weaknesses; (2) describe the outdoor localization technologies, and compare their strengths and weaknesses.  (8 points)

\item Design a conceptual system that includes the technologies you have learnt from this course (T-110.5150), for example, P2P networks, mobile sensing, CDN/CCN, IoT, etc. You may use diagram, UML chart, or any supplementary texts to elaborate. (10 points)


\end{enumerate}

\section*{Notes}

\begin{itemize}
\item Please answer the questions concisely in English.
\item Your own ideas about the technology are highly appreciated.
\item Plagiarism is strictly forbidden.
\item Please add a reference if you use texts from the Internet or elsewhere.
\item Format: typed, A4 size, font size 10, 1.5 line spacing.
\item Length: Q1-Q4 maximum two pages each, Q5 and Q6 maximum four pages.
\end{itemize}

\end{document}

