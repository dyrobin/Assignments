\documentclass[12pt, a4paper]{article}
\usepackage{hyperref}
\usepackage{indentfirst}
\usepackage[pdftex]{graphicx}
\usepackage{verbatimbox}
\usepackage{placeins}
\title{Peer-to-Peer Network}
\author{T-110.5150 - Applications and Services in Internet}
\date{Deadline: 31st of October 2013, 23:59 Helsinki Time}
\begin{document}
\maketitle
\section{Description}
In this assignment, you will form a small team (maximum 2 members) to implement a peer-to-peer (P2P) application based on a simple protocol.
The major functionalities of this application include:

\begin{itemize}
\item Join the experimental P2P network for this assignment.
\item Search for a key in the P2P network.
\item Publish a key from your node and make it visible to other nodes in the P2P network.
\end{itemize}

In order to test and observe the behaviour of P2P nodes, an experimental P2P network is constructed.
This network is Gnutella style (a simplified version), without a central facility.
Each node uses a piece of bootstrap information to join the network.
In order to communicate with other nodes, your application should fulfil the requirements of the protocol.

Details of the protocol are specified in the next section.
Grading is based on numbers of features your application provides, the final report, and the demo.
See the grading section for more information.


\section{Protocol Specification and Behaviours}
The protocol for this assignment is a binary protocol and originated from Gnutella 0.6 \cite{gnutella}.
This protocol works on top of TCP and contains a 16 bytes fixed header.
Similar to other binary network protocols, all value fields are in network byte order.
In order to differentiate from TCP packets, we use the term ``message'' in our protocol.

\subsection{Protocol header}
\begin{verbbox}
0                          16                        32
+------------+------------+------------+------------+
|  Version   |     TTL    | Msg Type   |  Reserved  |
+------------+------------+------------+------------+
|       Sender Port       |       Payload length    |
+------------+------------+------------+------------+
|            Original Sender IP Address             |
+------------+------------+------------+------------+
|                     Message ID                    |
+------------+------------+------------+------------+
\end{verbbox}

\begin{figure}[h!]
  \centering
  \theverbbox
  \caption{Protocol header}
    \label{fig:header}
\end{figure}

Figure \ref{fig:header} illustrates the structure of the protocol header.
The meaning of different fields are as follow:

\textbf{Version}: The version of this protocol, always one.
Any message with other version values MUST be dropped.

\textbf{TTL}: Time To Live.
This field MUST be less than or equal to five.
Each time when a message is forwarded, this field MUST be decreased by one.
If TTL of a message equals to zero, this message MUST be discarded.

\textbf{Msg Type}: The types of message, valid values are:
\begin{itemize}
\item 0x00 Ping
\item 0x01 Pong
\item 0x02 Bye
\item 0x03 Join
\item 0x80 Query
\item 0x81 Query Hit
\end{itemize}

\textbf{Original Sender IP Address}: The IPv4 address of the original sender of this message.
The intermediate nodes MUST NOT change this field when forwarding the message.

\textbf{Sender Port}: The listening port number of the original sender.
The intermediate nodes MUST NOT change this field when forwarding the message.

\textbf{Message ID}: The message ID SHOULD be globally unique for each message. 
One way to generate the ID is to put the IP of the sender, port, time stamp, and a sequence number together, then hash them to create a message ID.

\textbf{Payload length}: The length of payload in bytes.
The header length is NOT included.

\textbf{Reserved}: Set to zero. Unused in this version.

\subsection{Types of message}
\subsubsection{0x03 Join}
There are two sub-types of Join message: Join Request and Join Response.
Join Request has no payload.

A Join Response has a 2 bytes body, which contains the result of Join Request.
A new node sends a Join Request message to a node that is already in the network for the permission of entering the network.
This is the so-called bootstrap process.
A node which has already joined the network MAY also send a Join Request to another node to expand its own neighbourhood.
By connecting with more nodes, the reliability is enhanced.

A Join Response has a 2 bytes body, which contains the result of Join Request.
If the Join Request is accepted by a remote peer, it MUST respond with a Join Response message with the same message ID and a 0x0200 status code in the message body.
After that, both nodes consider each other as a normal peer and keep the TCP connection for message exchange.

In version one of our protocol, there is only one valid status code (0x0200).
All others are invalid and how to handle them are implementation dependant.

To refuse a Join Request, a standard way is to close the TCP connection.

A node SHOULD be able to connect to any given IP and port combination.

\begin{verbbox}
0                          16
+------------+------------+
|      Status code        |
+------------+------------+
\end{verbbox}

\begin{figure}[h!]
  \centering
  \theverbbox
  \label{joinrequest}
  \caption{Join message}
\end{figure}

\begin{verbbox}
0                          16
+------------+------------+
|    0x02    |    0x00    |
+------------+------------+
\end{verbbox}

\begin{figure}[h!]
  \centering
  \theverbbox
  \label{joinresponse}
  \caption{Join response (acceptance)}
\end{figure}

\subsubsection{0x00 Ping}
There are two purposes for a Ping message:
\begin{itemize}
\item Type A: Availability test (heart-beat).
\begin{itemize}
\item A Ping message with TTL equals to 1.
\item It is recommended to send a Type A Ping message to your neighbours every 5 seconds.
\item A node receiving a Type A Ping message SHOULD respond with a Type A Pong message.
\item If neither Ping or Pong message is received from a peer for some time, a node MAY assume that there is some problem with this peer.
However, the node SHOULD still keep this connection, in case the peer does not support Ping and Pong message.
\end{itemize}
\item Type B: Network probe
\begin{itemize}
\item A Ping message with TTL greater than 1.
\item This function aims to find more peers based on existing nodes.
\item A node receiving a Type B Ping message SHOULD respond with a Type B Pong message. 
\end{itemize}
\end{itemize}

Ping messages carry NO payload.
The only difference between Ping type A and B is the TTL field.

\subsubsection{0x01 Pong}
There are two types of Pong messages.

\begin{itemize}
\item Type A: A Pong message responding to an availability test.
This Pong message carries no payload.
\item Type B: A Pong message responding to a network probe.
This Pong message contains maximum five entries of the node's neighbours, excluding the one which sent the Ping message.
A node may have more than five entries available, how to choose among them is implementation dependant.
Although a good selection strategy greatly enhances the robustness of the network, you may choose randomly or just the first five for simplicity.
\end{itemize}

Since a Pong message responds to a Ping message, its message ID is identical to that of the corresponding Ping message.

\begin{verbbox}
0                         16                        32
+------------+------------+------------+------------+
|        Entry size       |           SBZ           |
+------------+------------+------------+------------+
|               IP Address of the 1st Entry         |
+------------+------------+------------+------------+
|   Port of 1st Entry     |    	      SBZ           |
+------------+------------+------------+------------+
|               IP Address of the 2nd Entry         |
+------------+------------+------------+------------+
|   Port of 2nd Entry     |    	      SBZ           |
+------------+------------+------------+------------+
|   (the list continues)                            |
+------------+------------+------------+------------+
\end{verbbox}

\begin{figure}[h!]
  \centering
  \theverbbox
  \label{pong}
  \caption{Pong message}
\end{figure}

\textbf{Entry size}: The number of entries in the Pong message body.

\textbf{SBZ}: Should be zero. These bits are reserved.

\textbf{IP address of Xnd Entry}: The IP address of Xnd Entry.

\textbf{Port of Xnd Entry}: The listening port number of Xnd Entry.

\subsubsection{0x80 Query}
A Query Message contains a query key in the message body.
It is propagated in the network according to its TTL.
A node receiving this message SHOULD look up its own sharing data, and return the matched entries, if any, in a Query Hit message.
It SHOULD also forward this message to its neighbours, if TTL is greater than one, no matter the query is a hit or not among its local shared files.

The length of search criteria varies. The actual length is determined by the Payload length field in the message header.
Nodes MUST NOT change the message ID of a Query Message while forwarding the message.

A node SHOULD keep a record of recently received Query Messages and their senders or forwarders.
Note that there may be multiple paths between two nodes.
If a node receives a Query Message whose message ID is identical to a recent Query, the message SHOULD be discarded.

\begin{verbbox}
+------------+------------+------------+------------+
|     Search criteria (variable length)             |
+------------+------------+------------+------------+
\end{verbbox}

\begin{figure}[h!]
  \centering
  \theverbbox
  \label{query}
  \caption{Query message}
\end{figure}


\subsubsection{0x81 Query Hit}
The message ID of a Query Hit message is identical to that of the corresponding Query message.
The body of Query Hit message contains entries of matched resources.
Each entry provides a resource ID and the corresponding resource value.
The first 2 bytes of a Query Hit message body contains the total number of entries in the message.

A Query Hit message is routed back to the Query sender using the reverse path of the Query message.

\begin{verbbox}
0                          16                        32
+------------+------------+------------+------------+
|        Entry size       |           SBZ           |
+------------+------------+------------+------------+
|    Resource ID (1st)    |           SBZ           |
+------------+------------+------------+------------+
|                Resource Value (1st)               |
+------------+------------+------------+------------+
|    Resource ID (2nd)    |           SBZ           |
+------------+------------+------------+------------+
|                Resource Value (2nd)               |
+------------+------------+------------+------------+
|    (the list continues)                           |
+------------+------------+------------+------------+
\end{verbbox}
\begin{figure}[h!]
  \centering
  \theverbbox
  \label{queryhit}
  \caption{Query Hit}
\end{figure}

\textbf{Entry size}: The number of entries in the Query Hit message body.

\textbf{SBZ}: Should be zero.

\textbf{Resource ID (Xnd)}: The ID of the resource.

\textbf{Resource Value}: The value of the resource.

\subsubsection{0x02 Bye}
Bye message is optional with TTL equals to 1.
The receiver MUST NOT forward this message and MUST close the TCP connection immediately.
The sender SHOULD wait for the termination of the connection.
A Bye message carries no payload.

\section{Detailed Instructions and Tips}
\subsection*{Which development environment and programming language can I use?}
We highly recommend you to use C under Linux/Unix environment.
If you are not familiar with network programming, learning it in C enables full grasp of the network programming concepts.
On the contrary, high level languages usually come with highly wrapped libraries which hide those details from you.

One alternative is Java.
However, we can only provide the protocol specification and sample codes in C.
If you use Java, you have to adapt them by yourself.

\subsection*{Network Byte Order}
The P2P protocol is a binary protocol, and like other network protocol, it follows network byte order.
The network byte order is big-endian.
However, your workstation is most likely of x86 architecture.
In other words, it is little-endian.
Keep the endianness \footnote{\url{http://en.wikipedia.org/wiki/Endianness}} in your mind.

\subsection*{How to bootstrap?}
Information regarding the bootstrap network will be announced later through Noppa.
The bootstrap information contains the IP address and the listening port of existing nodes in the network.
A new node can send a Join Message to an IP and port combination to participate in the network.
We will also provide some test keys so you can search for them in the network.

\subsection*{Should I implement the full protocol specification in this assignment?}
Implementation of the whole protocol is not mandatory.
Only fulfilling parts of them is enough to pass the assignment.
However, you are encouraged to explore further.

\subsection*{Should I consider NAT or firewall problems?}
You are not required to traverse NAT or firewall in this assignment.
In the Gnutella v0.6 protocol, there is a Push message to mitigate these problems.
Read the Gnutella RFC for more information.

Note, however, that there might be an NAT or a firewall in your working environment.
They may hamper the interaction between your nodes and the bootstrap network or other students' nodes.

\subsection*{Should I provide a GUI?}
The purpose of this assignment is not a fancy GUI.
However, if you want to try GUI programming, feel free to do it.

\subsection*{Should I use multi-thread/process?}
The assignment is designed so that it is achievable with a single process.
The different demo task may be implemented as separate applications, as long as you achieve the goals.
An API you might be interested in is \texttt{select()}, which is available in both C and Java.

Nevertheless, you are encouraged to explore different possibilities.

\subsection*{How to locate the problem in my code?}
For network programming, \emph{Wireshark} or \emph{tcpdump} are useful for checking the contents of the packets you send to and receive from the network.
Observe the inbound and outbound packets to check if you are actually sending or receiving the messages and whether they conforms to the format of the protocol.

The common debugging tool for Linux is \emph{GDB}, which allows you to set break points in you programme and step through the program.
There is also similar debugger for Java, and can be easily launched in IDE such as Eclipse or NetBeans.

Printing out the states often helps you to track down the problem.

\subsection*{Think before coding}
Conquer the feature list one by one without having a full view is discouraged.
Eventually you may find that you have to rewrite your application in order to implement the next feature.
Understand all features and have a clear design increase your productivity.


\section{Submission}
Each team will be provided a Niksula Git \footnote{\url{https://git.niksula.hut.fi/}} repository for version control and submission.
Please check also Git documentation \footnote{\url{http://git-scm.com/doc}} if you have not used it before.

You should create an \texttt{assignment\_1} directory in the repository.
The contents of the directory are:
\begin{verbatim}
assignment_1/
  src/
  README
  Report.pdf
\end{verbatim}

\begin{itemize}
\item The \texttt{src} directory contains all your source code.
\item Submit only the source code, not binary executables.
\item The \texttt{README} file is a text file and it should explain how to build and run your application.
\item Clearly state your solution for this assignment in \texttt{Report.pdf}. Please be concise and go directly to the main points. The report should be no more than 4 pages long.
\item Please follow the directory structure and naming convention, otherwise your submission will be ignored.
\item If you have other files that need to be included, please explain why you need them in the \texttt{README} file.
\end{itemize}

Deadline is strict and commits after the deadline will not be graded.
Please plan early your workload and test early the features which you are implementing.

Frequent commits to Git repository is encouraged.
Try not to send only one big commit in the end, but many small commits as the development proceeds.

Unfortunately, we cannot guarantee 100\% uptime in the bootstrap servers.
Should they become unavailable please send us an e-mail so we can fix it as soon as possible.


\section{Demo}
The demo session is 30 minutes sharp, which includes 5 minutes for setup, 20 minutes for presenting the application, and 5 minutes for questions.
You must check out the source code from the Git and build your application from source onsite during the demo session.
Both team members shall be present at the demo session.
The basic requirements are:
\begin{itemize}
\item Show the application is able to search for a given key in the network, and retrieve the corresponding value(s).
\item Show the application is able to publish a given key to the network.
\end{itemize}
You are free to decide how to present other features of your application, but remember to make a plan and finish your demo on time.
Detailed arrangements of the demo session will be announced later.


\section{Grading}
The assignment is split into several functions, and points are given separately to each function.
To pass the assignment, you have to implement all features in the basic level.
Basic level features also satisfy the basic requirements of demo sessions.

\vskip 20pt

fail --- $<$ 20 points \emph{or} a missing basic feature.

1 --- $\ge$ 20 points.

2 --- $\ge$  40 points.

3 --- $\ge$ 60 points.

4 --- $\ge$ 80 points.

5 --- $\ge$  100 points.

\vskip 20pt

Total: 130 Points

Please check the attached tables to see the detailed grading criteria.

\begin{table}[htdp]
\caption{Basic (20 points)}
\begin{center}
\begin{tabular}{|p{12cm}|p{2cm}|}
\hline
Send Join Request message in order to join the network & 5 Points \\
\hline
Handle Join Request message. Consider the situation that remote peer closes the connection. & 5 Points\\
\hline
Send Query message with a search key & 5 Points  \\
\hline
Publish a key/value pair in the network and respond to a Query message when its search key equals your published key & 5 Points \\
\hline
\end{tabular}
\end{center}
\end{table}

\begin{table}[htdp]
\caption{Advanced (80 points)}
\begin{center}
\begin{tabular}{|p{12cm}|p{2cm}|}
\hline
Respond to Type A Ping message with a Pong.
Tell your neighbour that you are still alive. & 10 points \\
\hline
Send Type A Ping message.
Check if your neighbours are alive. & 10 Points \\
\hline
Respond to Type B Ping message with a Pong.
Tell your neighbour who else you know. & 10 Points \\
\hline
Send Type B Ping message to find more node information in the network & 10 Points \\
\hline
Join multiple peers in order to expand your network & 10 Points \\
\hline
Forward Query message with loop avoidance to help other node to find a resource & 15 Points \\
\hline
Forward Query Hit message by following the reverse path of its corresponding Query message & 15 Points \\
\hline
\end{tabular}
\end{center}
\end{table}

\begin{table}[htdp]
\caption{Other (30 points)}
\begin{center}
\begin{tabular}{|p{12cm}|p{2cm}|}
\hline
A clear final Report.
State how you did the assignment.
Comments on the assignment are also welcomed.
No more than 4 pages and do not paste source code. & 10 points \\
\hline
A good demo.
Show good understanding of the assignment, application runs well, and finish the demo within allotted time. & 10 points \\
\hline
Extra points for constructive comments on this assignment, a good implementation, or other unexpected but decent outcome. & 10 points \\

\hline
\end{tabular}
\end{center}
\end{table}


\FloatBarrier

\begin{thebibliography}{1}

\bibitem{gnutella}
Gnutella Protocol v0.6, \url{http://rfc-gnutella.sourceforge.net/src/rfc-0\_6-draft.html}.

\end{thebibliography}


\end{document}
